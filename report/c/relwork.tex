Most of the research that has been conducted on intersection management systems for autonomous vehicles tends to prefer centralised systems that can handle traffic requests.
Particularly relevant is the research of two major groups in this area: the researchers at the University of Texas at Austin, and an international group of researchers composed by members of the Massachusetts Institute of Technology (MIT), the Swiss Institute of Technology (ETHZ), and the Italian National Research Council (CNR).\\

At the University of Texas, Kurt Dresner and Peter Stone developed AIM, Autonomous Intersection Management, a reservation-based system built around a detailed communication protocol able to coordinate movement of self driving cars through intersections \cite{texas}.
Through a simulation they are able to demonstrate the potential of this system to outperform current intersection control mechanism: traffic lights and stop signs.
In the simulation, the intersection center is divided into a n x n grid of reservation tiles.
Through a "first come, first served" policy, approaching vehicles make a request to the system to reserve the space-time they need to cross the intersection. 
Their trajectory is then computed by the system and if the requesting vehicle at any time occupies a reservation tile that is already in use, the request is rejected.
The vehicle will then continue requesting until it can pass.

A similar approach is also followed by the international group of MIT, ETHZ and CNR researchers.
They developed a centralised slot-based intersection \cite{mit}.
In their simulations, cars adjust their velocities in approaching the intersection in order to arrive and cross at a given slot of time that is made available for them.
This system also involves communication between vehicles and a centrlised controller.