The reactive approach can be seen as a stepping stone to achieve autonomy.
The nature of the controller makes more advanced schemes more complex to incorporate.
Our controller is an example of how far you can get with subsumption architectures - we did collision avoidance and built right hand rule onto that.

We ran multiple simulations changing factors that affected the performance significantly and commented on the results.

While our systems performance is comparable to AIM\cite{texas}, it is nowhere near as good.
In fact, the AIM system performs up to three times better in terms of throughput and doesn't produce neither deadlocks nor collisions.

Centralised systems can naturally achieve much more optimisation than pure reactive ones.

These results also imply that we have room for improvement.
With additional work put into the controller and the sensory layer, it should be achievable to design a more advanced version that doesn't produce collisions nor deadlocks, or at least reduces the occurrences significantly.

Despite the fact that our system performs poorly compared to others, its strength is the omission of centralised control. 
Centralised control introduces a construct prone to single point of failure.

The vision was that pure reactive control should provide a safe fallback system in case centralised control fails.
%way to traverse intersections with less sources for errors.
