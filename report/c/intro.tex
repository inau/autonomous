%A presentation of previous research (related work), different paradigms of control (basic theory), and our angle.

The idea of having autonomous agents has existed in science fiction literature for hundreds of years and can also be traced back to ancient mythology.
The ideas and concepts of autonomous agents are hence not something entirely new, but rather something that has been refined over the course of a lot of years.
However the technology has just recently reached a state where a lot of the underlying challenges to autonomous agents can be solved somewhat efficiently.
Self driving cars are becoming more and more a reality, and they will be an important presence in the near future.
By automising vehicle control, traffic management can arguably be made more efficient in many ways, starting with throughput but also in other aspects, like fuel management and emissions control. 

In this paper we will focus on what it takes to make autonomous automobiles efficient in passing intersections.

Initially we had a look at what other researchers had been doing the recent decade and quickly realised that there was a pattern. 
The same groups keep appearing in the literature when having the focus on autonomous vehicles and intersections.

To put everything in context we will start by presenting some of the challenges that are present when trying to efficiently solve the task of managing vehicles in an intersection.
Followed by different control paradigms. These paradigms will be presented in relation to existing solutions to match the theory to their practical implementation.
%emphasize how practical applications of these abstract paradigms can look.  
