Reactive vs. deliberate vs. hybrid.

\textbf{maybe theory makes more sense before related work}\\

The theory we have touched upon is from multiple fields of engineering. One field is within traffic engineering, namely intersection management. The other is within the field of robotics, namely robot control.

We use the term agents for robots or other actors moving within an environment. We use the term environment as being synonimous to the world in which the agent is confined.

\subsection{Reactive Control}
This paragdim is also coined Sense-Act. This is due to the fact that it can be explained as simply as sensing the environment and reacting instantly to the registered values.
Consider a simple robot with only one distance sensor in the front. While there is nothing infront of it for a certain distance it moves forward, else it does not.

\subsection{Deliberate Control}
This paragdim is also known as Sense-Plan-Act. Compared to the reactive control it introduces a layer of planning. This however requires alot more information about the environment. Any change to the environment would also invalidate the current plan as this is not factored in. This is computationally heavy since it senses, plans a step, executes the step and starts over.
A plan can consist of multiple steps which are required to reach a goal.

\subsection{Hybrid Control}
This paragdigm is a hybrid of the two previous control paragdigms. 
