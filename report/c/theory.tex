Reactive vs. deliberate vs. hybrid.

\textbf{maybe theory makes more sense before related work}\\

The theory we have touched upon is from multiple fields of engineering. One field is within traffic engineering, namely intersection management. The other is within the field of robotics, namely robot control.

We use the term agents for robots or other actors moving within an environment.\\
We use the term environment as being synonimous to the world in which the agent is confined.
The following will contain short presentations of the three canonical ways of doing robot control.

\subsection{Reactive Control}
This paragdim is also coined Sense-Act. This is due to the fact that it can be explained as simply as sensing the environment and reacting instantly to the registered values.
Consider a simple robot with only one distance sensor in the front and the goal of avoiding collisions. While there is nothing infront of it for a certain distance it moves forward, else it does not.

\subsection{Deliberate Control}
This paragdim is also known as Sense-Plan-Act.
Compared to the reactive control it introduces a layer of planning.
This however requires alot more information about the environment.
Any change to the environment would also invalidate the current plan as this is not factored in.
This is computationally heavy since it senses, plans a step, executes the step and starts over.
A plan can consist of multiple steps which are required to reach a goal.
Consider a simple robot as the one previously mentioned needing to go from point A to B.
While its tempting to just point it in the direction and hope it succeeds, any obstacle in the way would make this difficult.
However translating the goal to multiple steps of moving in a direction, and handling the case of being unable to perform a step by invalidating the current plan and recalculating it, makes this approach better suited for complex tasks.

\subsection{Hybrid Control}
This paragdigm is a hybrid of the two previous control paragdigms. 
It can be explained as decomposing a complex goal into sub tasks. Lets say a robot needing to pickup a object and moving it to a position.
It has one behaviour for gripping the object once the sensor has identified it, and another for actually going from A to B.
