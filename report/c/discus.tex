\begin{itemize}
\item Discuss experimental results. ~
\item Shortcomings of model.
\item Improvements for further tests.
\item traffic fairness - distribution
\end{itemize}

  \subsection{shortcomings}
 
As discussed in the previous section, the system has some clear shortcomings.
First of all, safety is not ensured since there is still occurrence of \textbf{collisions} between vehicles.
According to our observations though, most of the collisions happened in deadlock or semi-deadlock situations in which the cars were moving extremely slowly trying to pass through, which is a condition that can be easily corrected.
Another case in which collisions were relatively frequent was when cars approached the intersection at high speeds.
In 1 out of 10 tests, a collision happened in the first minute of the simulation.
Our system can handle collision avoidance in most cases, but with specific angles of impact and speed there is not enough brake distance to react in time.
\newline

Another shortcoming of our system is the abundant presence of deadlocks.
This condition is harder to solve, because it is caused by an issue underlying in the structure of the intersection.
Cars trying to pass through the same space concurrently can easily block each other, especially without a centralised controller that can sort the traffic.

\subsection{improvements and further work}
 %sensory layer, obstacle avoidance, mimic locks with pure reactive (don't go if anyone is inside, safer, probably slower, but not in the long run)\\
%roundabouts, other paradigms that are less complex

Here we discuss about how some of the issues encountered could be approached and how further work could be structured.

To improve safety of the system, further improvements of the sensory layer could be performed.
The current model is rather simplistic in its implementation, especially comparing it to real life car sensor systems.
