\documentclass[titlepage]{article}

%packages
%\usepackage{biblatex}
\usepackage[numbers]{natbib}
\usepackage{titlepic}
\usepackage{graphicx}

\title{Autonomous Car Intersection Simulation Project}
\author{
   Ivan Naumovski\\
    {inau@itu.dk}\\
    \\
   Martino Secchi\\
    {msec@itu.dk}\\
}
\titlepic{\includegraphics[scale=.3]{img/itulogo.jpg}}
\date{December 2016}


%\addbibresource{report2.bib}
\begin{document}

\begin{titlepage}
	\centering

	{\huge\bfseries Autonomous Car Intersection Simulation Project \par}
	\vspace{2cm}
	{\Large Ivan Naumovski, {inau@itu.dk}\\ Martino Secchi, {msec@itu.dk}\\\par}
	\vfill
	supervised by\par
	\large Prof.~Kasper \textsc{St\o y}

	\vfill
	\includegraphics[width=0.3\textwidth]{img/itulogo.jpg}\par\vspace{1cm}
% Bottom of the page
	{\large \today\par}
\end{titlepage}

\pagenumbering{gobble}
%\maketitle
\clearpage
\tableofcontents
\clearpage

\pagenumbering{arabic}

\abstract{
\begin{em}
This project aims to investigate some of the aspects involving traffic management with autonomous vehicles.
In particular, we want to know how it can be possible to regulate self driving cars through an intersection, and if it is possible to achieve this without a centralised controller and the use of communication.
We built a simulation environment using the Unity3D game engine. We tried different representations of intersections and designed a reactive controller.

We gathered data describing throughput, collisions and deadlocks.
%The controller we built performs worse than the compared centralised solution from Texas University.
%The strength is that by going pure reactive we avoid having a single point of failure planning mechanism.
The system we developed doesn't achieve the same performance of centralised solutions, 
but we will compare results and discuss about the different approaches.
The main goal of the project is the development of a system that is purely reactive, without the single point of failure of centralised solutions.
Our solution is intended as a fallback for when systems relying on communication fail.
\end{em}
}
\section{Introduction}
A presentation of previous research (related work), different paragdigms of control (basic theory), and our angle.

\subsection{Motivation}
Reactive vs. deliberate vs. hybrid.

\textbf{maybe theory makes more sense before related work}\\

The theory we have touched upon is from multiple fields of engineering. One field is within traffic engineering, namely intersection management. The other is within the field of robotics, namely robot control.

We use the term agents for robots or other actors moving within an environment.\\
We use the term environment as being synonimous to the world in which the agent is confined.
The following will contain short presentations of the three canonical ways of doing robot control.

\subsection{Reactive Control}
This paragdim is also coined Sense-Act. This is due to the fact that it can be explained as simply as sensing the environment and reacting instantly to the registered values.
Consider a simple robot with only one distance sensor in the front and the goal of avoiding collisions. While there is nothing infront of it for a certain distance it moves forward, else it does not.

\subsection{Deliberate Control}
This paragdim is also known as Sense-Plan-Act.
Compared to the reactive control it introduces a layer of planning.
This however requires alot more information about the environment.
Any change to the environment would also invalidate the current plan as this is not factored in.
This is computationally heavy since it senses, plans a step, executes the step and starts over.
A plan can consist of multiple steps which are required to reach a goal.
Consider a simple robot as the one previously mentioned needing to go from point A to B.
While its tempting to just point it in the direction and hope it succeeds, any obstacle in the way would make this difficult.
However translating the goal to multiple steps of moving in a direction, and handling the case of being unable to perform a step by invalidating the current plan and recalculating it, makes this approach better suited for complex tasks.

\subsection{Hybrid Control}
This paragdigm is a hybrid of the two previous control paragdigms. 
It can be explained as decomposing a complex goal into sub tasks. Lets say a robot needing to pickup a object and moving it to a position.
It has one behaviour for gripping the object once the sensor has identified it, and another for actually going from A to B.


\section{Considerations}
Post-research we were in a limbo, since the landscape of tooling that can be used for setting up experiments in this field of studies is  quite  vast.
However it was decided early in the proces that it would be mainly software simuation, due to the fact that robots have a higher cost and getting the amount of agents that makes these types of studies interesting is somewhat higher than a dozen of robots.

\subsection{Simulation Tools}
Performing simulations is not trivial. Different criteria should be met depending on the accuracy of the simulation that is to be performed.
In particular one should decide what level of abstraction fits the scenario, as 1:1 simulations are significantly harder to model, but might not necessarilly add value matching the increased complexity.\\

This introduces the question, what modelling complexity is sufficient for modelling our problem?\\

While the real world has three dimensions, we only need a two dimensional model.
The height dimension is not required since we can consider every agent to occupy a space in two dimensional space.
Restricting the simulated world to only two axis makes the model alot simpler to model.
If this was a simulation for aircrafts the height dimension would make a lot more sense to include.\\
%Another point is that we need the capability of being able to express physical shapes and collisions between these shapes.\\
%being able to apply kinematic forces to these shapes.

Two tools have been considered for doing the simulated world. One was the GazeboSimulator, which is a widely adopted tool in the robotics community.
Another was doing a simulation using game development tools, in particular Unity3D.

While both tools provide the functionality we need such as physics simulation in 2D/3D  environments, the degree of tools supported out-of-the-box varies.
Gazebo has robotic simulations as being its main function, and hence has alot of useful tools for robotics integrated into it.
Unity3D on the other hand is a general-purpose tool so it requires some tailoring to get the same functionality as Gazebo.

Unity3D however is a strong competitor due to the fact that we have prior experience using this tool. It is easy to go from concept to an actual prototype rapidly. And last but not least has a thriving community of professionals, academics and hobbyists aspiring to make the software great.\\
Our solution is built using Unity3D.\\

The choice, albeit it might seem arbitrary, is based on the fact that \textit{Gazebo} has a very steep learning curve.


\section{Implementation}
\begin{figure}
\centering
\includegraphics[scale=.6]{img/classdiagram}
\caption{Overview of Unity3D software package}
\label{figure:classdiagram}
\end{figure}

Post-research we were in a limbo, since the landscape of tooling that can be used for setting up experiments in this field of studies is  quite  vast.
However it was decided early in the proces that it would be mainly software simuation, due to the fact that robots have a higher cost and getting the amount of agents that makes these types of studies interesting is somewhat higher than a dozen of robots.

\subsection{Simulation Tools}
Performing simulations is not trivial. Different criteria should be met depending on the accuracy of the simulation that is to be performed.
In particular one should decide what level of abstraction fits the scenario, as 1:1 simulations are significantly harder to model, but might not necessarilly add value matching the increased complexity.\\

This introduces the question, what modelling complexity is sufficient for modelling our problem?\\

While the real world has three dimensions, we only need a two dimensional model.
The height dimension is not required since we can consider every agent to occupy a space in two dimensional space.
Restricting the simulated world to only two axis makes the model alot simpler to model.
If this was a simulation for aircrafts the height dimension would make a lot more sense to include.\\
%Another point is that we need the capability of being able to express physical shapes and collisions between these shapes.\\
%being able to apply kinematic forces to these shapes.

Two tools have been considered for doing the simulated world. One was the GazeboSimulator, which is a widely adopted tool in the robotics community.
Another was doing a simulation using game development tools, in particular Unity3D.

While both tools provide the functionality we need such as physics simulation in 2D/3D  environments, the degree of tools supported out-of-the-box varies.
Gazebo has robotic simulations as being its main function, and hence has alot of useful tools for robotics integrated into it.
Unity3D on the other hand is a general-purpose tool so it requires some tailoring to get the same functionality as Gazebo.

Unity3D however is a strong competitor due to the fact that we have prior experience using this tool. It is easy to go from concept to an actual prototype rapidly. And last but not least has a thriving community of professionals, academics and hobbyists aspiring to make the software great.\\
Our solution is built using Unity3D.\\

The choice, albeit it might seem arbitrary, is based on the fact that \textit{Gazebo} has a very steep learning curve.

\subsection{Software components}
As we have settled on using Unity3D we work within its boundaries.
It is component driven, meaning every object in the world consists of a collection of components and a transform - the transform is the objects spatial information.
These components can be either colliders, renderers, or custom scripts, and objects can contain other objects - they can be composite.

The implementation was done by splitting the work in three major blocks. One was \textit{the vehicle model}, another was \textit{the intersection model} and the last \textit{the reactive controller} for the car.
An overview of our software package is depicted on figure \ref{figure:classdiagram}.

%\begin{itemize}
%\item Presentation of our model.
%\item Intersection w/wo traffic lights. Cars going straight vs. cars turning.
%\item Car model and sensor modelling (sick vs simplified directional).
%\item Graph for navigation.
%\item Reactive controller.
%\item Special rules (right hand).
%\item Sensor range limitation based on braking distance considerations.
%\item SimScale to RealWorldScale
%\end{itemize}

\subsection{The World Model}
As observed in the real world, multiple intersection models exist, three lane and 4 lane intersections are the most common ones.
The style can be either with the roundabout approach, which essentially rules out the left turn complication or a traditional 4 lane intersection where going left crosses the opposing traffics lane.

In our model we mainly focus on a single lane 4 way intersection.\\ We tried different intersection representations as can be seen on figures \ref{figure:graph} and \ref{figure:graphs}.

%\textbf{insert simple intersection images - maybe graph images from unity}
\begin{figure}
\centering
\includegraphics[scale=.4]{img/graph.png}
\caption{Image of the underlying graph representation for a four way inersection.}
\label{figure:graph}
\end{figure}

\begin{figure}
\centering
\includegraphics[height=75px]{img/graph-old1.png}
\includegraphics[height=75px]{img/graph-old2.png}
\includegraphics[height=75px]{img/graph-old3.png}
\caption{Image of the previous versions of the underlying graph representation.}
\label{figure:graphs}
\end{figure}

%We also consider a 4 way roundabout.

We represent these intersection models using an underlying directed graph.
 A car agent in our environment is seeded with information about its origin and its destination.
These points are vertices in our graph. 
Every combination of origin and destination pair has one path.

While this does not provide alot of flexibility in straying from the path, the simplicity makes the model really easy to implement and adjust.

Once we reached the test phase we also realised that we needed to change the values to real world values.

We translated our cars to being a fixed meter value. This value was not arbitrarly chosen, it was based of values from the range in which cars of that class belong to, namely compact car.
Compact category covers cars such as Ford Focus, lengths are between 4.1 and 4.45 meters for hatchbacks and a bit more for sedans, saloons and station wagons.
The simulation car size was 0.8 units. We went with 4.4 meters as being the real world value, and hence got a scaling factor of $5.5$.
Using the scaling factor we also translated the simulated vehicles speeds to km/h, which is quite helpful for the comparisons.

\subsubsection{The traffic lights}
This is an attempt to mimic traffic light regulated intersections.
The lights change at intervals chosen in regards to our experiences with intersections. 
These values can vary alot from intersection to intersection - and are highly influenced by traffic flow during the day. 
We did experiments with different interval windows.

\begin{figure}
\centering
\includegraphics[scale=.2]{img/tesla_sensor}
\caption{Image of the sensors integrated into a car.}
\label{figure:tesla_sens}
\end{figure}

\subsection{The Car Model}
The way the car has been modelled is using a interface which essentially contains actions such as accellerating or decellerating, braking, and turning left or right.
This abstraction simplifies the step of going from a virtual model to a physical one.
The interface provides loose coupling between the controller and the underlying vehicle implementation.
This seperation makes it simple to develop multiple types of controllers or vehicles.

\begin{figure}
\centering
\includegraphics[scale=.5]{img/sensors}
\caption{Image of Unity sensor model.}
\label{figure:unity_sens}
\end{figure}

\subsubsection{The sensors}
Studying autonomous vehicles, such as the Tesla's we realised how much sensing is actually integrated in these marvels.
As seen on figure \ref{figure:tesla_sens}, a lot of sensors exist for achieving autonomous control in vehicles.
The figure is not representative for all autonomous vehicles - but can be seen as a good starting point for understanding the complexity of the vehicles.\\

We focused on mimicking the frontal sensors and the cross-traffic sensors.
We investigated the products of a well-known vendor, SICK, which is used by many different companies within the field.
The sensing is done similar to the current generation of autonomous vehicles. We mimick 190 degree LMS5xx series for the frontal sensor. This is a sensor which uses lasers. It provides a high degree of precision. It functions in a lot of different conditions and works in the range of 0 to 80 meters.

\subsection{The Reactive Controller}
This was developed incrementally. Initial versions were done using only collision avoidance.
The vision was to have multiple different controllers and be able to hotswap these for the simulation.
Later versions also included constructs for priority such as the right hand rule and a finer grained distinction between moving objects and their movement in relation to the agent.\\

Due to the simplification of only using one frontal sensor compared to the amount of sensors others might use, we have a finer grained way to interpret the input from this abstraction.
The 180 degreee cone is split into zones, which essentially mimics the same division of the frontal senses as seen on figures \ref{figure:tesla_sens} and \ref{figure:unity_sens}.
We took the abstraction further and limited the range of detection based on velocity of the vehicle. The faster it goes the farther it looks ahead.
This reduces the risk of vehicles being overly cautious and stopping in intersections with lots of slow moving vehicle, especially if it is still possible to pass through.
Additionally the braking distance is not as big at lower velocities - which makes this approach viable.\\

The controller has been reiterated multiple times: initially it was only doing collision avoidance.
This was achieved by acting directly on the frontal sensor values.
The earliest versions only had one threshold. If anything was closer than the given threshold the car would not move.
While this avoids colliding with anything infront, it does not handle following a car infront very well.
We did another version which trails behind the car in front.
This is done using the delta distance rather than raw distance - if both vehicles move at the same speed the delta stays zero.
If the gap closes it gains a negative value and if the gap increases it gains a positive value.


\section{Experiments}
%we have some fixed parameters such as light control on or off. Mimics intersection lights.\\
%Whether simulated vehicles are able to turn or not.\\
%We have different max speeds - 60 or 40 km per hour approach. This translates to lower speeds %during turns (roughly half).\\

%tests are different combinations of above parameters. And results are measured in collisions, cars spawned, cars reaching destination, and time before deadlock.\\
%Deadlock is the situation where cars have been stuck for a period of time in the `intersection zone' without entering or leaving.

%To test the implementation of our system we ran some simulations 
To compare performances of the system over different paradigms and with different parameters, we ran
multiple experimental simulations.
The parameters that we changed over the different tests are from three classes: the paradigm, the speed of the cars, and the complexity.
For paradigm it is meant the use of  a traffic light regulated intersection versus a pure reactive approach.
As for complexity, we can increase it by allowing the cars to turn in order to reach any destination, or we can use a simpler system where cars can only go straight.
The car speed is changed over three base cases: slow ( 30 km/h), medium (40 km/h) and fast (60 km/h).
It is worth mentioning that the advised speed when turning the car in 90 degrees turns inside urban areas is around or below 30 km/h. %todo find reference

When simulating with traffic lights enabled, we test two intervals of time that regulate the duration of the green light.
We have an interval window of approximately 1 minute, that reflects a realistic scenario according to our observation of real life intersections.
We also test a much shorter window of 20 seconds, that should improve performance under the assumptions that:
\begin{enumerate}
\item autonomous vehicles can react faster than human drivers, so they can start moving immediately after the traffic light turns green compared to human drivers that could be slow or distracted;
\item shorter windows of time allow the traffic to be managed evenly ... %continue
\end{enumerate}

\section{Discussion}
Discuss experimental results.
Shortcomings of model.
Improvements for further tests.

\section{Conclusion}
The reactive approach can be seen as a stepping stone to achieve autonomy.
The nature of the controller makes more advanced schemes more complex to incorporate.
Our controller is an example of how far you can get with subsumption architectures - we did collision avoidance and built right hand rule onto that.

While our systems performance is comparable to AIM, it is nowhere near as good.
In fact, the AIM system performs up to three times better in terms of throughput and doesn't produce neither deadlocks nor collisions.
%The throughput is lower. Our system can deadlock.
This naturally implies that we have room for improvement.
With additional work put into the controller, it should be achievable to design a controller which does not collide.

Despite the fact that it performs poorly compared to other systems, its strength is the omission of centralised control. 
Centralised control introduces a construct prone to single point of failure.

The vision was that pure reactive control should provide a safe fallback system in case centralised control fails.
%way to traverse intersections with less sources for errors.


\bibliographystyle{unsrtnat}
\bibliography{report2}
%\printbibliography

\end{document}